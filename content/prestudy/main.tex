\documentclass[stu, 12pt, a4paper, donotrepeattitle, floatsintext]{apa7}
\usepackage[ngerman]{babel}
\usepackage[utf8]{inputenc}
\usepackage[T1]{fontenc}
\usepackage{csquotes}
\usepackage{graphicx}
\graphicspath{{../resources/images/}{../resources/}}
\usepackage{hyperref}
\usepackage{pifont} % For checkmarks and other symbols
\usepackage{tabularx}

\usepackage{threeparttable}
\usepackage{booktabs}
\usepackage[ngerman]{translator}
\usepackage{pgfgantt}
\usepackage{ragged2e} % For better ragged text in tables
\usepackage{pdflscape}
\usepackage{float} % Allow forcing float placement
\usepackage[style=apa,sortcites=true,sorting=nyt,backend=biber]{biblatex}
\DeclareLanguageMapping{ngerman}{ngerman-apa}

% Enable numbered sections (apa7-native, no titlesec needed)
\setcounter{secnumdepth}{4}

\usepackage{subcaption}
\usepackage{subfiles} % Best loaded last in the preamble

% Force German names if apa7 overrides them
\AtBeginDocument{\renewcommand{\abstractname}{Zusammenfassung}}

\addbibresource{../resources/bibliography.bib}

\title{Vertrauen in Künstliche Intelligenz \\ Wie Framing das Vertrauen in LLM-basierte Applikationen- und Antworten beeinflusst \\ Vorstudie Bachelorarbeit}
\shorttitle{}
\author{Fabian Ryf}
\affiliation{{Hochschule Luzern, Wirtschaft} \\ HSLU-W}
\course{BSc Business Psychology (BP) \\ Markt- und Konsumentenpsychologie}
\professor{Dr. Andreas Hüsser}
\duedate{05.12.2025}

\begin{document}
\newcommand{\isMain}{1}

% Start page numbering from the title page
\pagenumbering{arabic}

\maketitle

\setcounter{tocdepth}{3}
\tableofcontents
\newpage

% Reset counters for clean sequential numbering
\setcounter{table}{0}
\setcounter{figure}{0}

% Content continues with existing page count (no reset)

% ============================================
% SECTION 1: Einleitung
%   1.1 Ausgangslage
%   1.2 Zielsetzung
%   1.3 Machbarkeit
%   1.4 Arbeitsplan
% ============================================
\subfile{sections_required/01_einleitung}

% ============================================
% SECTION 2: Theoretische Einbettung
%   (with subsections from 02_theory)
% ============================================
\subfile{sections/02_theory}

% ============================================
% SECTION 3: Forschungsfrage
% ============================================
\newpage
\subfile{sections_required/02_forschungsfrage}

% ============================================
% SECTION 4: Forschungsdesign  
%   (with subsections from 03_methodology)
% ============================================
\subfile{sections/03_methodology}

% ============================================
% SECTION 4: Selbstreflexion und Ausblick
%   4.1 Selbstreflexion
%   4.2 Weiteres Vorgehen
% ============================================
\subfile{sections_required/04_selbstreflexion}

% ============================================
% SECTION 5: Quellenverzeichnis (Bibliography)
% ============================================
\newpage
\subfile{sections_required/10_quellenverzeichnis}

% ============================================
% SECTION 6: Kommentiertes Literaturverzeichnis
% ============================================
\newpage
\subfile{sections/08_commented_review}

% ============================================
% SECTION 7: Abbildungsverzeichnis
% ============================================
\newpage
\subfile{sections_required/11_abbildungen}

% ============================================
% SECTION 8: Tabellenverzeichnis
% ============================================
\newpage
\subfile{sections_required/12_tabellenverzeichnis}

% ============================================
% SECTION 9: Glossar
% ============================================
\newpage
\subfile{sections_required/13_glossar}

% ============================================
% SECTION 11: Anhang
% ============================================
\vspace{1em}
\section{Anhang}

\end{document}
