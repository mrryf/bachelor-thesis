\documentclass[../main.tex]{subfiles}
\begin{document}
\section{Theoretische Einbettung}

Modelltheoretisch knüpft die vorliegende Arbeit an frühere Studien in den Bereichen Vertrauen in künstliche Intelligenz/technologische Veränderungen, wahrgenommene Nützlichkeit sowie Benutzerfreundlichkeit und die daraus abgeleitete Nutzungsabsicht an. Als theoretische Grundlage dient zunächst das Technology Acceptance Model, welches die Rahmenbedingungen zur Analyse von Adoptionsprozessen neuer Technologien schafft \parencite{davis_perceived_1989}. Den zweiten Baustein liefert die Erweiterung des TAM-Modells durch \textcite{baroni_ai-tam_2022}. Diese ergänzt das Modell um zusätzliche Faktoren wie das Vertrauen in KI-gestützte Assistenten und bildet diese im Artificial Intelligence Technology Acceptance Model (AI-TAM) ab \parencite{baroni_ai-tam_2022}. Zuletzt wird der Framing-Effekt theoretisch beleuchtet, da dieser für die gewählte Stimulus-Wahl relevant ist. Konkret wird dabei die Form des Attribute-Framing-Effekts betrachtet \parencite{druckman_evaluating_2001, freling_when_2014}.

\subsubsection[Technology Acceptance Model]{Technology Acceptance Model \parencite{davis_perceived_1989}}
Das TAM wurde entwickelt, um die mangelnde Nutzerakzeptanz von Informationssystemen zu adressieren, die als Haupthindernis für den Erfolg neuer Technologien identifiziert wurde. Davis untersuchte 112 Angestellte und Manager eines grossen nordamerikanischen Unternehmens, die zwei unterschiedliche Softwaresysteme nutzten - ein elektronisches Mailsystem und einen Texteditor. Das Modell basiert auf der Attitude-Paradigm aus der Psychologie, speziell auf Fishbein und Ajzens Theory of Reasoned Action \parencite{fishbein_belief_1975}. TAM besagt, dass die tatsächliche Systemnutzung durch die Verhaltensintention bestimmt wird, welche von der Einstellung zur Nutzung abhängt.

Diese Einstellung wird durch zwei zentrale Konstrukte geprägt: Perceived Usefulness, definiert als «the degree to which an individual believes that using a particular system would enhance his or her job performance», sowie Perceived Ease of Use, verstanden als «the degree to which an individual believes that using a particular system would be free of physical and mental effort». Die Studie zeigte, dass Perceived Usefulness etwa 50\% einflussreicher auf die Nutzung war als Ease of Use, wobei das Modell 36\% der Varianz in der tatsächlichen Nutzung erklären konnte.



\paragraph{Erweiterungen des TAM zum AI-TAM}
\textcite{baroni_ai-tam_2022} erweiterten das TAM um drei zusätzliche Konstrukte: «Explainable AI Trust» (Vertrauen in KI) aus der Literatur zu «Explainable AI» (XAI), «Collaborative Intention» (Kollaborationsabsicht) zur Messung der Bereitschaft zur Teilnahme an «Human-in-the-Loop»-Mechanismen sowie die Vertrautheit mit der Technologie und dem Anwendungskontext. Das im AI-TAM verwendete Vertrauenskonstrukt entstammt der Forschung von \textcite{hoffman_metrics_2019} und erfasst, inwieweit Nutzer den Ergebnissen eines KI-Systems vertrauen. Ergänzend misst die Kollaborationsabsicht die Bereitschaft, aktiv an der Weiterentwicklung der KI mitzuwirken.

\begin{figure}[ht]
    \centering
    \includegraphics[width=\textwidth, height=0.8\textheight, keepaspectratio]{ba_faryf_ai_tam.jpg}
    \caption{Erweitertes TAM-Modell: Artificial Intelligence-Technology Acceptance Model \parencite{baroni_ai-tam_2022}}
    \label{fig:ai-tam-model}
\end{figure}



\paragraph{Framing-Theorie als Stimulus-Konzept}
Die Verbindung des AI-TAM mit dem Konzept des Framings eröffnet neue Forschungsperspektiven. Besonders das Attribute Framing könnte das XAIT-Konstrukt beeinflussen: Die Präsentation von KI-Fähigkeiten mit unterschiedlichen Konfidenz-Stufen (hoch, mittel, tief) dürfte das Vertrauen in die KI direkt verändern. Gemäss dem AI-TAM-Modell beeinflusst dieser Faktor wiederum die Nutzungsabsicht. Für Experimente mit Large Language Models bedeutet dies, dass die Art der Leistungsdarstellung die Nutzerakzeptanz beeinflussen könnte. Das AI-TAM bietet hierbei den methodischen Rahmen, um diese Effekte präzise zu messen. Im Folgenden wird das Konzept des Attribute Framings näher erläutert.

\subsubsection{Framing-Effekt}
Der Framing-Effekt, erstmals von Kahneman und Tversky in ihrer Prospect Theory beschrieben, zeigt, dass Entscheidungen davon beeinflusst werden, wie Informationen präsentiert werden \parencite{tversky_framing_1986}. Der Framing-Effekt zeigt unter anderem, wie identische Szenarien zu unterschiedlichen Präferenzen führen, je nachdem ob sie in Gewinn- oder Verlustbegriffen formuliert werden. Während sich die frühe Forschung auf riskante Entscheidungen konzentrierte, erweiterte sich das Konzept auf verschiedene Framing-Typen wie Risky Choice Framing, Goal Framing und Attribute Framing.

Zusätzlich untersuchte \textcite{freling_when_2014} in ihrer Meta-Analyse 107 Studien zum Attribute Framing und entwickelten dabei eine theoretische Integration mittels Construal Level Theory (CLT). Ihre zentrale Erkenntnis: Die Effektivität von Attribute Framing hängt von der Kongruenz zwischen dem Abstraktionsniveau (Construal Level) des Frames und der psychologischen Distanz des Bewertenden zum geframten Event ab \parencite{freling_when_2014}.

\paragraph{Attribute Framing nach Levin \& Gaeth (1988) und \textcite{dolgopolova_effect_2022}}
Attribute Framing unterscheidet sich von anderen Framing-Typen, da hier ein einzelnes Attribut in äquivalenten aber unterschiedlich valenten Begriffen beschrieben wird. Levin und Gaeth demonstrierten dies mit Hackfleisch, das entweder als «75\% mager» oder «25\% fett» beschrieben wurde (Levin \& Gaeth, 1988). Der Attribute Framing-Effekt manifestiert sich in einer valenz-konsistenten Verschiebung: Positive Frames führen zu günstigeren Bewertungen als negative. Ihre Studie zeigte zudem, dass direkte Produkterfahrung den Framing-Effekt abschwächt - ein Befund, der durch ein Averaging-Modell erklärt wird, bei dem zusätzliche Informationsquellen den relativen Einfluss einzelner Frames reduzieren.

\textcite{dolgopolova_effect_2022} fanden bei Lebensmittelentscheidungen differenzierte Effekte: Gain-Frames erzeugten positivere Einstellungen, jedoch keinen signifikanten Effekt auf Kaufintentionen. Framing-Effekte fallen somit je nach abhängiger Variable unterschiedlich aus. Der Attribute Framing-Effekt ist für die KI-Akzeptanz relevant, da KI-Systeme durch unterschiedliche Konfidenz-Darstellungen charakterisiert werden können. Das AI-TAM bietet den Rahmen, um diese Effekte auf Vertrauen und Nutzungsabsicht zu untersuchen.

\subsection{Forschungsfrage}
Wie beeinflusst die Framingdarstellung von KI-Konfidenzwerten (positiv vs. negativ) das Vertrauen in KI-generierte Antworten und die daraus resultierende Technologieakzeptanz in LLM-basierten Assistenzsystemen?

\subsection{Latente Konstrukte}
Die latenten Konstrukte werden mittels einer Online-Befragung nach der Nutzung der KI-Assistenz erhoben. Die verwendeten Konstrukte basieren auf dem AI-TAM von \textcite{baroni_ai-tam_2022} und wurden aus drei Quellen adaptiert: \textcite{ibrahim_technology_2025} für die Kernkonstrukte wahrgenommene Nützlichkeit, Einfachheit der Nutzung, Verhaltensintention und Vertrauen in KI; \textcite{topsakal_how_2025} für die technologische Vorerfahrung; sowie \textcite{grassi_knowledge-grounded_2022} für die Kollaborationsintention. Alle Items werden auf einer 5-stufigen Likert-Skala erhoben.

\begin{table}[ht]
    \centering
    \caption{Identifizierte latente Konstrukte zum Einsatz von AI-TAM}
    \label{tab:latent-constructs}
    \begin{tabularx}{\textwidth}{l p{4cm} X p{4cm}}
        \toprule
        \textbf{Abkürzung} & \textbf{Name} & \textbf{Definition} & \textbf{Quelle} \\
        \midrule
        XAIT & Explainable AI trust & Vertrauen in die generierte Antwort und die LLM-Lösung & \parencite{baroni_ai-tam_2022} \\
        BI & Behavorial Intention & Verhaltensintention die LLM-Lösung zu Nutzen & \parencite{baroni_ai-tam_2022, davis_perceived_1989} \\
        CI & Collaborative Intention & Kollaborationsintention zur digitalen Assistenz & \parencite{baroni_ai-tam_2022, davis_perceived_1989} \\
        PUF & Perceived Usefulness & Wahrgenommene Nützlichkeit der generierten Antwort & \parencite{baroni_ai-tam_2022, davis_perceived_1989} \\
        EOU & Ease of Use & Einfachheit in der Nutzung der KI-Applikation & \parencite{baroni_ai-tam_2022, davis_perceived_1989} \\
        FAM-TEC & Familiarity with Technology & Vertrautheit in der Nutzung von KI-Technologie & \parencite{baroni_ai-tam_2022, davis_perceived_1989} \\
        \bottomrule
    \end{tabularx}
\end{table}

\subsection{Hypothesenübersicht}
\begin{table}[ht]
    \centering
    \caption{Aufgelistete Hypothesen im Rahmen der Bachelor-Arbeit-Vorstudie}
    \label{tab:hypotheses}
    \begin{tabularx}{\textwidth}{l X c l}
        \toprule
        \textbf{Hypothese} & \textbf{Pfad} & \textbf{Richtung} & \textbf{Theorie} \\
        \midrule
        H1a & Dummy\_Pos (Stimulus) $\to$ XAIT & + & Attribute Frame \\
        H1b & Dummy\_Neg (Stimulus) $\to$ XAIT & - & Attribute Frame \\
        H2 & XAIT $\to$ PUF & + & AI-TAM-Modell \\
        H3 & XAIT $\to$ EOU & + & AI-TAM-Modell \\
        H4 & PUF $\to$ BI & + & AI-TAM-Modell \\
        H5 & EOU $\to$ BI & + & AI-TAM-Modell \\
        H6 & EOU $\to$ PUF & + & AI-TAM-Modell \\
        H7 & BI $\to$ CI & + & AI-TAM-Modell \\
        H8 & FAMTEC $\to$ PUF & + & AI-TAM-Modell \\
        \midrule
        \multicolumn{4}{l}{\textbf{Kovariate}} \\
        ACTS & ACTS $\to$ XAIT & kontrolliert & - \\
        \midrule
        \multicolumn{4}{l}{\textbf{Mediation}} \\
        H9 & Framing $\to$ XAIT $\to$ (PUF, EOU) $\to$ BI & Indirekt & - \\
        \bottomrule
    \end{tabularx}
    \par\medskip
    \footnotesize{Anmerkung: Alle Pfade werden simultan im Strukturgleichungsmodell (SEM) geschätzt}
\end{table}

\subsubsection{Ausformulierte Hypothesen}

\paragraph{Haupthypothesen (Framing-Effekte)}
\begin{itemize}
    \item H1a: Die positive Darstellung des Accuracy Scores (z.B. «Diese Antwort ist zu 80\% korrekt») führt zu einem höheren AI Output Trust als die Kontrollbedingung ohne Score-Anzeige.
    \item H1b: Die negative Darstellung des Accuracy Scores (z.B. «Diese Antwort hat eine 20\% Fehlerwahrscheinlichkeit») führt zu einem niedrigeren AI Output Trust als die Kontrollbedingung ohne Score-Anzeige.
\end{itemize}

\paragraph{AI-TAM Kernbeziehungen}
\begin{itemize}
    \item H2: AI Output Trust hat einen positiven Einfluss auf die Perceived Usefulness. Nutzer, die den AI-Ausgaben vertrauen, bewerten das System als nützlicher für ihre Aufgaben.
    \item H3: AI Output Trust hat einen positiven Einfluss auf die Perceived Ease of Use. Vertrauen in die AI-Ausgaben reduziert die wahrgenommene kognitive Belastung bei der Systemnutzung.
\end{itemize}

\paragraph{TAM-Standardbeziehungen}
\begin{itemize}
    \item H4: Die Perceived Usefulness hat einen positiven Einfluss auf die Behavioral Intention. Je nützlicher Nutzer Alva einschätzen, desto höher ist ihre Absicht, das System zukünftig zu nutzen.
    \item H5: Die Perceived Ease of Use hat einen positiven Einfluss auf die Behavioral Intention. Eine als einfach wahrgenommene Nutzung erhöht die Intention zur zukünftigen Systemnutzung.
    \item H6: Die Perceived Ease of Use hat einen positiven Einfluss auf die Perceived Usefulness. Systeme, die einfach zu nutzen sind, werden als nützlicher wahrgenommen.
    \item H7: Die Behavioral Intention hat einen positiven Einfluss auf die Collaborative Intention. Nutzer, die beabsichtigen Alva zu nutzen, zeigen auch eine höhere Bereitschaft zur kollaborativen Zusammenarbeit mit dem AI-System.
    \item H8: Familiarity with Technology hat einen positiven Einfluss auf die Perceived Usefulness. Nutzer, die mit KI-Technologie vertraut sind, schätzen die Nützlichkeit der digitalen Assistenz höher ein.
\end{itemize}

\paragraph{Mediation}
\begin{itemize}
    \item H9: Der Effekt des Framings auf die Behavorial Intention wird durch Explainable AI Trust partiell oder vollständig mediiert. Bei niedrigen Explainable AI Trust-Werten ist der Unterschied zwischen positivem und negativem Framing grösser als bei hohen Explainable AI Trust-Werten.
\end{itemize}

\begin{figure}[ht]
    \centering
    \includegraphics[width=\textwidth, height=0.8\textheight, keepaspectratio]{ba_faryf_messmodell.jpg}
    \caption{Strukturmodell für AI-TAM, Teil des Strukturgleichungsmodell, inkl. Hypothesen, inkl. Stimulus}
    \label{fig:structure-model}
\end{figure}
\end{document}
