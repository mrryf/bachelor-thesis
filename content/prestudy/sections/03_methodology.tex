\documentclass[../main.tex]{subfiles}
\begin{document}
\section{Forschungsdesign}
Das Experiment untersucht den Einfluss von Framing bezüglich Sicherheit und Unsicherheit auf das Vertrauen in KI-Systeme. In einem 3x2 Between-Subjects-Design wird die Darstellung von Konfidenzwerten (Sicherheit vs. Unsicherheit) bei variierenden Accuracy-Scores (hoch, mittel, niedrig) manipuliert. Daraus ergeben sich sechs Experimentalgruppen sowie eine Kontrollgruppe ohne Score-Anzeige. Die experimentelle Manipulation erfolgt während der realen Interaktion mit einem KI-Assistenten.

\subsection{Experimentelles Design}

Das Untersuchungsdesign entspricht einem 3x2 faktoriellen Between-Subjects-Design. Die erste unabhängige Variable (Framing) variiert die Darstellung als Sicherheit versus Unsicherheit, die zweite unabhängige Variable (Accuracy) variiert den Konfidenzwert in drei Stufen (hoch, mittel, niedrig). Dieses Design ermöglicht die Untersuchung von Haupt- und Interaktionseffekten beider Faktoren auf die abhängigen Variablen

\begin{table}[ht]
    \centering
    \caption{Experiment-Design, vorhandene Experimentalbedingungen}
    \label{tab:experiment-design}
    \begin{tabularx}{\textwidth}{l l X X}
        \toprule
        \textbf{Bedingung} & \textbf{Gruppe} & \textbf{Manipulation} & \textbf{Beispiel} \\
        \midrule
        Positive-Frame & Gruppe 1 & Score wird als Konfidenz/Zuverlässigkeit dargestellt & «Antwortsicherheit: 80\%» oder «Antwortsicherheit zu 80\% zuverlässig» \\
        Negative-Frame & Gruppe 2 & Score wird als Unsicherheit/Fehlerwahrscheinlichkeit dargestellt & Antwortunsicherheit: 20\% oder «Diese Antwort hat eine Fehlerwahrscheinlichkeit von 20\%» \\
        Kontrollgruppe & Gruppe 3 & Kein Score wird angezeigt (Status Quo) & - \\
        \bottomrule
    \end{tabularx}
\end{table}

\begin{table}[ht]
    \centering
    \begin{threeparttable}
        \caption{Experimentelles 3x2 Design: Manipulation von Framing und Accuracy}
        \label{tab:experiment-design-3x2}
        \begin{tabularx}{\textwidth}{l X X}
            \toprule
             & \multicolumn{2}{c}{\textbf{Framing (Unabhängige Variable 1)}} \\
            \cmidrule(lr){2-3}
            \textbf{Accuracy (UV 2)} & \textbf{Positiver Frame} & \textbf{Negativer Frame} \\
            \midrule
            \textbf{Hoch} & Sicherheit: Hoch & Unsicherheit: Tief \\
            \textbf{Mittel} & Sicherheit: Mittel & Unsicherheit: Mittel \\
            \textbf{Niedrig} & Sicherheit: Tief & Unsicherheit: Hoch \\
            \bottomrule
        \end{tabularx}
        \begin{tablenotes}
            \small
            \item \textit{Anmerkung.} Die Kontrollgruppe (kein Score) ist in diesem 3x2 Design nicht abgebildet.
        \end{tablenotes}
    \end{threeparttable}
\end{table}

\subsection{Ablauf Experiment}
Das geplante Experiment findet in drei Phasen statt. In der ersten Phase werden die Nutzenden über das Experiment informiert und können sich für oder gegen eine Teilnahme entscheiden. In Phase 2 steht die Interaktion mit dem Chatbot Alva im Zentrum. In Phase 3 werden die Nutzenden aufgefordert, die dazugehörige Umfrage auszufüllen und das Experiment abzuschliessen. Es werden keine Daten im Vorfeld (Phase 1) oder während der Interaktion (Phase 2) erhoben, um die Abbruchrate zu minimieren und eine hohe Abschlussrate zu fördern.

\subsection{Stimulus-Konzept}
Der Stimulus besteht aus der visuellen und textlichen Darstellung des AI-Accuracy Scores direkt nach jeder LLM-Antwort. Die Manipulation erfolgt in Echtzeit während der natürlichen Interaktion mit dem digitalen Assistenten.

% Tabelle 4: Spezifikation Stimuli

\begin{figure}[htbp]
    \centering
    \begin{subfigure}[b]{0.45\textwidth}
        \centering
        \includegraphics[width=\textwidth]{design/ba_fryf_stimulus_design/sicherheit_hoch.pdf}
        \caption{Stimulus: Sicherheit hoch}
        \label{fig:stimulus_sicherheit_hoch}
    \end{subfigure}
    \hfill
    \begin{subfigure}[b]{0.48\textwidth}
        \centering
        \includegraphics[width=\textwidth]{design/ba_fryf_stimulus_design/unsicherheit_tief.pdf}
        \caption{Stimulus: Unsicherheit tief}
        \label{fig:stimulus_unsicherheit_tief}
    \end{subfigure}
    \caption{Stimulus-Design: Hohe Sicherheit vs. Tiefe Unsicherheit}
    \label{fig:stimulus_design_hoch_tief}
\end{figure}

\begin{figure}[h!]
    \centering
    \begin{subfigure}[b]{0.48\textwidth}
        \centering
        \includegraphics[width=\textwidth]{design/ba_fryf_stimulus_design/sicherheit_mittel.pdf}
        \caption{Stimulus: Sicherheit mittel}
        \label{fig:stimulus_sicherheit_mittel}
    \end{subfigure}
    \hfill
    \begin{subfigure}[b]{0.48\textwidth}
        \centering
        \includegraphics[width=\textwidth]{design/ba_fryf_stimulus_design/unsicherheit_mittel.pdf}
        \caption{Stimulus: Unsicherheit mittel}
        \label{fig:stimulus_unsicherheit_mittel}
    \end{subfigure}
    \caption{Stimulus-Design: Mittlere Sicherheit vs. Mittlere Unsicherheit}
    \label{fig:stimulus_design_mittel}
\end{figure}

\begin{figure}[h!]
    \centering
    \begin{subfigure}[b]{0.48\textwidth}
        \centering
        \includegraphics[width=\textwidth]{design/ba_fryf_stimulus_design/sicherheit_tief.pdf}
        \caption{Stimulus: Sicherheit tief}
        \label{fig:stimulus_sicherheit_tief}
    \end{subfigure}
    \hfill
    \begin{subfigure}[b]{0.48\textwidth}
        \centering
        \includegraphics[width=\textwidth]{design/ba_fryf_stimulus_design/unsicherheit_hoch.pdf}
        \caption{Negativer Frame / Niedrige Sicherheit}
        \label{fig:stimulus-neg-low}
    \end{subfigure}
    
    \caption{Übersicht der Stimulus-Designs für das 3x2 Experiment}
    \label{fig:stimulus-overview}
\end{figure}

\subsubsection{Manipulationscheck Stimulus}
Als Manipulationscheck werden die Probanden post-experimentell gefragt, ob und in welcher Form ihnen Informationen zur Zuverlässigkeit der Antworten angezeigt wurden, um sicherzustellen, dass die experimentelle Manipulation wahrgenommen wurde.

\subsection{Methodentriangulation}
Das vorliegende Forschungsdesign kombiniert verschiedene Methoden in einer Methodentriangulation, um die Forschungsfrage nach dem Einfluss von KI-Transparenz auf Vertrauen zu beantworten.

Die Triangulation erfolgt auf drei Ebenen:
\begin{itemize}
    \item experimentelle Manipulation des Accuracy Framings als Between-Subject-Design gewährleistet die interne Validität durch randomisierte Zuweisung.
    \item Die natürliche Beobachtung während der tatsächlichen LLM-Nutzung erhöht die ökologische Validität, da Nutzer eigene Fragen in realistischen Anwendungskontexten stellen.
    \item Die standardisierte Befragung mittels validierter Skalen aus dem AI-TAM-Modell ermöglicht die reliable Messung latenter Konstrukte.
\end{itemize}

\subsubsection{Methodenintegration}
Die Integration der Datenquellen erfolgt auf Analyseebene: Die experimentelle Gruppenzugehörigkeit (Framing und Konfidenz-Level) wird mit den Befragungsdaten (Vertrauen, TAM-Konstrukte) in einem gemeinsamen Datensatz zusammengeführt. Diese Integration ermöglicht:
\begin{itemize}
    \item Haupteffekte: Der Einfluss von Framing (Sicherheit vs. Unsicherheit) und Konfidenz-Level (hoch, mittel, tief) auf das Vertrauen kann separat analysiert werden
    \item Interaktionseffekte: Die Wechselwirkung zwischen Framing und Konfidenz-Level kann untersucht werden
    \item Pfadanalysen: Die Beziehungen zwischen Vertrauen und den TAM-Konstrukten gemäss dem AI-TAM-Modell können geprüft werden
\end{itemize}



\subsection{Abgrenzung des Forschungsdesigns}
Die vorliegende Studie fokussiert auf die valenzorientierte Darstellung von KI-Leistungsmetriken (Attribute Framing) und deren Einfluss auf Vertrauen und Technologie-akzeptanz im Kontext des AI-TAM-Modells.

\subsubsection{Inhaltlich}
\begin{itemize}
    \item Andere Framing-Typen: Die Studie beschränkt sich auf Attribute Framing und untersucht nicht Risky Choice Framing oder Goal Framing.
    \item Langzeiteffekte (keine Längsschnittstudie): Gemessen wird die Nutzungsabsicht (Behavioral Intention), nicht die tatsächliche Systemnutzung über längere Zeiträume. Der Intention-Behavior-Gap wird nicht untersucht.
    \item Alternative Transparenzmechanismen: Neben der Score-Darstellung existieren weitere Transparenzmöglichkeiten (Erklärungen, Quellenangaben, Visualisierungen), die nicht Gegenstand dieser Arbeit sind.
    \item Kontextübergreifende Generalisierung: Die Untersuchung findet im Verwaltungskontext statt. Ob die Ergebnisse auf medizinische, kreative oder andere Anwendungsbereiche übertragbar sind, bleibt offen.
\end{itemize}

\subsubsection{Methodisch}
\begin{itemize}
    \item Between-Subjects-Design: Jede Person erfährt nur eine Framing-Bedingung. Intraindividuelle Vergleiche sind nicht möglich.
    \item Quantitative Fokussierung: Die Studie nutzt standardisierte Skalen, verzichtet jedoch auf qualitative Vertiefungen wie Interviews zur Exploration der zugrunde liegenden kognitiven Prozesse.
    \item Natürliche Variation des Accuracy Scores: Der tatsächliche Score wird nicht experimentell manipuliert, sondern variiert basierend auf den Nutzeranfragen. Er dient als Kovariate, nicht als unabhängige Variable.
\end{itemize}

\end{document}
