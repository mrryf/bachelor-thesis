\documentclass[../main.tex]{subfiles}
\begin{document}
\section{Forschungsdesign}
Das Experiment untersucht den Einfluss von Framing bezüglich Sicherheit und Unsicherheit auf das Vertrauen in KI-Systeme. In einem 3x2 Between-Subjects-Design wird die Darstellung von Konfidenzwerten (Sicherheit vs. Unsicherheit) bei variierenden Accuracy-Scores (hoch, mittel, niedrig) manipuliert. Daraus ergeben sich sechs Experimentalgruppen sowie eine Kontrollgruppe ohne Score-Anzeige. Die experimentelle Manipulation erfolgt während der realen Interaktion mit einem KI-Assistenten.

\subsection{Experimentelles Design}

Das Untersuchungsdesign entspricht einem 3x2 faktoriellen Between-Subjects-Design. Die erste unabhängige Variable (Framing) variiert die Darstellung als Sicherheit versus Unsicherheit, die zweite unabhängige Variable (Accuracy) variiert den Konfidenzwert in drei Stufen (hoch, mittel, niedrig). Dieses Design ermöglicht die Untersuchung von Haupt- und Interaktionseffekten beider Faktoren auf die abhängigen Variablen

\begin{table}[ht]
    \centering
    \caption{Experiment-Design, vorhandene Experimentalbedingungen}
    \label{tab:experiment-design}
    \begin{tabularx}{\textwidth}{l l X X}
        \toprule
        \textbf{Bedingung} & \textbf{Gruppe} & \textbf{Manipulation} & \textbf{Beispiel} \\
        \midrule
        Positive-Frame & Gruppe 1 & Score wird als Konfidenz/Zuverlässigkeit dargestellt & «Antwortsicherheit: 80\%» oder «Antwortsicherheit zu 80\% zuverlässig» \\
        Negative-Frame & Gruppe 2 & Score wird als Unsicherheit/Fehlerwahrscheinlichkeit dargestellt & Antwortunsicherheit: 20\% oder «Diese Antwort hat eine Fehlerwahrscheinlichkeit von 20\%» \\
        Kontrollgruppe & Gruppe 3 & Kein Score wird angezeigt (Status Quo) & - \\
        \bottomrule
    \end{tabularx}
\end{table}



\subsection{Stimulus-Konzept}
Der Stimulus besteht aus der visuellen und textlichen Darstellung einer Sicherheits- bzw. Unsicherheitsanzeige, die direkt nach jeder LLM-Antwort entsprechend der zugewiesenen Stimulusgruppe eingeblendet wird. Die Manipulation erfolgt in Echtzeit während der natürlichen Interaktion mit dem digitalen Assistenten.

Das Stimulus-Design setzt sich aus zwei Dimensionen zusammen. Die erste Dimension betrifft die Valenz der Darstellung: Die Anzeige wird entweder positiv als „Sicherheit" oder negativ als „Unsicherheit" gerahmt. Die zweite Dimension umfasst die Ausprägungsstufe, wobei die angezeigte Sicherheit bzw. Unsicherheit in drei Stufen variiert – hoch, mittel und niedrig. Aus der Kombination dieser beiden Dimensionen ergibt sich ein 2×3-Design mit insgesamt sechs unterschiedlichen Stimulus-Bedingungen, die jeweils eine spezifische visuelle und textliche Gestaltung aufweisen.

\begin{table}[H]
    \centering
    \begin{threeparttable}
        \caption{Experimentelles 3x2 Design: Manipulation von Framing und Accuracy}
        \label{tab:experiment-design-3x2}
        \begin{tabularx}{\textwidth}{l X X}
            \toprule
             & \multicolumn{2}{c}{\textbf{Framing (Unabhängige Variable 1)}} \\
            \cmidrule(lr){2-3}
            \textbf{Accuracy (UV 2)} & \textbf{Positiver Frame} & \textbf{Negativer Frame} \\
            \midrule
            \textbf{Hoch} & Sicherheit: Hoch & Unsicherheit: Tief \\
            \textbf{Mittel} & Sicherheit: Mittel & Unsicherheit: Mittel \\
            \textbf{Niedrig} & Sicherheit: Tief & Unsicherheit: Hoch \\
            \bottomrule
        \end{tabularx}
        \begin{tablenotes}
            \small
            \item \textit{Anmerkung.} Die Kontrollgruppe (kein Score) ist in diesem 3x2 Design nicht abgebildet.
        \end{tablenotes}
    \end{threeparttable}
\end{table}

\subsubsection{Manipulationscheck Stimulus}
Als Manipulationscheck werden die Probanden post-experimentell gefragt, ob und in welcher Form ihnen Informationen zur Zuverlässigkeit der Antworten angezeigt wurden, um sicherzustellen, dass die experimentelle Manipulation wahrgenommen wurde.

\subsection{Ablauf Experiment}
Das geplante Experiment findet in drei Phasen statt. In der ersten Phase werden die Nutzenden über das Experiment informiert und können sich für oder gegen eine Teilnahme entscheiden. In Phase 2 steht die Interaktion mit dem Chatbot Alva im Zentrum. In Phase 3 werden die Nutzenden aufgefordert, die dazugehörige Umfrage auszufüllen und das Experiment abzuschliessen. Es werden keine Daten im Vorfeld (Phase 1) oder während der Interaktion (Phase 2) erhoben, um die Abbruchrate zu minimieren und eine hohe Abschlussrate zu fördern.

\begin{figure}[ht]
    \centering
    \includegraphics[width=\textwidth, keepaspectratio]{ba_faryf_ablauf_experiment.jpg}
    \caption{Ablauf des Experiments in drei Phasen}
    \label{fig:ablauf-experiment}
\end{figure}

\subsection{Methodische Einordnung des Forschungsdesigns}
Das vorliegende Forschungsdesign verbindet ein kontrolliertes Experiment mit einer Felderhebung im realen Nutzungskontext. Die Wahl dieser Methode orientiert sich an der Fragestellung und dem untersuchten Gegenstandsbereich \parencite[vgl.][S. 174f.]{kelle_mixed_2022}.

Die experimentelle Manipulation im Between-Subject-Design mit randomisierter Zuweisung zu den Experimentalbedingungen gewährleistet die interne Validität. Durch die Randomisierung wird sichergestellt, dass beobachtete Unterschiede in den abhängigen Variablen auf die experimentelle Manipulation (Framing und Konfidenz-Level) zurückgeführt werden können. Die Einbettung des Experiments in die tatsächliche Alva-Nutzung erhöht die ökologische Validität gegenüber rein laborbasierten oder szenariobasierten Designs. Proband*innen stellen eigene Fragen im realistischen Anwendungskontext, anstatt auf vorgegebene Szenarien zu reagieren. Die standardisierte Post-Befragung mittels validierter Skalen aus dem AI-TAM-Modell \parencite{baroni_ai-tam_2022} ermöglicht die reliable Messung der latenten Konstrukte.

\subsubsection{Methodenintegration}
Die Integration der Datenquellen erfolgt auf Analyseebene: Die experimentelle Gruppenzugehörigkeit (Framing und Konfidenz-Level) wird mit den Befragungsdaten (Vertrauen, TAM-Konstrukte) in einem gemeinsamen Datensatz zusammengeführt. Diese Integration ermöglicht die Analyse von Haupteffekten des Framings (Sicherheit vs. Unsicherheit) und des Konfidenz-Levels (hoch, mittel, tief) auf das Vertrauen, von Interaktionseffekten zwischen beiden Faktoren sowie von Pfadbeziehungen zwischen Vertrauen und den TAM-Konstrukten gemäss dem AI-TAM-Modell.



\subsection{Abgrenzung des Forschungsdesigns}
Die vorliegende Studie fokussiert auf die valenzorientierte Darstellung von KI-Leistungsmetriken (Attribute Framing) und deren Einfluss auf Vertrauen und Technologie-akzeptanz im Kontext des AI-TAM-Modells.

\begin{table}[H]
    \centering
    \caption{Inhaltliche Abgrenzung des Forschungsdesigns}
    \label{tab:inhaltliche-abgrenzung}
    \begin{tabularx}{\textwidth}{l X X}
        \toprule
        \textbf{Aspekt} & \textbf{Fokus dieser Studie} & \textbf{Abgrenzung} \\
        \midrule
        Framing-Typ & Attribute Framing & Risky Choice Framing, Goal Framing \\
        Zeithorizont & Nutzungsintention (einmalige Messung) & Tatsächliche Systemnutzung, Langzeiteffekte \\
        Transparenzmechanismen & Konfidenz-Darstellung & Erklärungen, Quellenangaben, Visualisierungen \\
        Anwendungskontext & Verwaltungskontext (Kanton Basel-Stadt) & Medizinische, kreative oder andere Bereiche \\
        \bottomrule
    \end{tabularx}
\end{table}

\begin{table}[H]
    \centering
    \caption{Methodische Abgrenzung des Forschungsdesigns}
    \label{tab:methodische-abgrenzung}
    \begin{tabularx}{\textwidth}{l X X}
        \toprule
        \textbf{Aspekt} & \textbf{Fokus dieser Studie} & \textbf{Abgrenzung} \\
        \midrule
        Studiendesign & Between-Subjects (3x2 faktoriell) & Within-Subjects, intraindividuelle Vergleiche \\
        Datenerhebung & Quantitativ (standardisierte Skalen) & Qualitative Vertiefungen, Interviews \\
        Konfidenz-Level & Experimentell manipuliert (hoch, mittel, tief) & Natürliche Variation \\
        \bottomrule
    \end{tabularx}
\end{table}

\end{document}
