\documentclass[../main.tex]{subfiles}
\begin{document}
\section{Ausgangslage}
Mit der Lancierung von Alva erweitert der Kanton Basel-Stadt sein bestehendes Informationsangebot um eine KI-gestützte Interaktionsform \parencite{kanton_basel-stadt_alva_2025, kanton_basel_stadt_2025}. Bei Alva handelt es sich um einen LLM-basierten Chatbot, der auf der Technologie von ChatGPT basiert und die Inhalte der Kantonswebsite bs.ch als Wissensbasis nutzt, um Fragen der Bevölkerung in natürlicher Konversationsform zu beantworten.

Die Einführung von Alva markiert für den Kanton Basel-Stadt einen bedeutsamen Schritt: Es handelt sich um eine der ersten KI-gestützten Lösungen dieser Art im kantonalen Kontext. Die gewonnenen Erkenntnisse aus diesem Pilotprojekt sollen als Grundlage für weitere KI-basierte Initiativen des Kantons dienen.

Aktuell verzeichnet Alva täglich rund 700 aktive Nutzer, die im Durchschnitt 1.4 Interaktionen mit dem digitalen Assistenten durchführen. Nebst der erwarteten Effizienzsteigerung bei der Informationsbeschaffung ist es von besonderem Interesse zu untersuchen, inwiefern das Vertrauen in die KI-Lösung die Nutzungsabsicht beeinflusst.
\end{document}
