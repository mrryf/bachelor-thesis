\documentclass[../main.tex]{subfiles}
\begin{document}
\section{Machbarkeit}

Die Vorstudie basiert auf einem experimentellen Testen verschiedener Konfidenz-Werte innerhalb eines chatbot-basierten Umfeldes. Das Experiment soll im Idealfall in einer tatsächlichen Chatbot-Interaktion stattfinden, anstelle einer fragebogenbasierten Stimulus-Darbietung.

Die von Basel-Stadt entwickelte Lösung «Alva» \parencite{alva_2025} dient als digitaler Assistent bei der Bedienung und Navigation der Website des Kantons Basel-Stadt \parencite{kanton_basel_stadt_2025}. Das Abrufen von Informationen funktioniert themen- und bereichsübergreifend, was inhaltlich anspruchsvolle Themen und Prozesse in einfache Schritte herunterbricht und die benötigten Links und Dokumente als Referenzinformatione. Alva zählt zum heutigen Zeitpunkt täglich rund 550 Nutzer mit durchschnittlich 1.4 Interaktionen pro Nutzer.

Nach initialen Unterhaltungen ist der Kanton Basel-Stadt einverstanden, das vorgesehene Experiment in der Live-Umgebung von Alva durchzuführen. Die anfallenden Arbeiten zur Integration werden zu je 50\% vom Auftraggeber Liip \parencite{liip_2025} und dem Kanton Basel-Stadt getragen. Die benötigte Stimulus-Konzeption und das Fragebogen-Design liegt in der Verantwortung des Studierenden.

\begin{table}[h!]
    \centering
    \caption{Meilensteine der Vorstudie und Bachelorarbeit}
    \label{tab:milestones}
    \begin{tabularx}{\textwidth}{@{}l l X@{}}
        \toprule
        \textbf{Meilenstein} & \textbf{Zeitraum} & \textbf{Beteiligte} \\
        \midrule
        Gespräch Machbarkeit intern & Juli 2025 & Liip \\
        Gespräch Machbarkeit extern & Oktober 2025 & Kanton Basel-Stadt \\
        Entwicklung Anforderungen (Logik \& Userflow) & Oktober 2025 & Studierender \\
        Schätzung benötigter Arbeiten & November 2025 & Product Owner, Frontend Developer \\
        Kommunikation Investment extern & November 2025 & Kanton Basel-Stadt \\
        Übereinkunft Investment-Teilung & November 2025 & Liip, Kanton Basel-Stadt \\
        \bottomrule
    \end{tabularx}
\end{table}


\end{document}
