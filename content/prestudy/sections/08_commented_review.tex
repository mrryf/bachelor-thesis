\documentclass[../main.tex]{subfiles}
\begin{document}
\section{Kommentiertes Literaturverzeichnis}

\fullcite{baroni_ai-tam_2022}

\vspace{1em}
Diese Studie entwickelt und validiert das AI Technology Acceptance Model (AI-TAM), eine Erweiterung des klassischen Technology Acceptance Models (TAM) für KI-gestützte Anwendungen mit Human-in-the-Loop-Mechanismen. Die Autoren integrieren Konstrukte aus der Explainable AI (XAI)-Literatur – insbesondere Nutzervertrauen in KI und wahrgenommene Qualität der KI-Ergebnisse – sowie das neue Konstrukt der Collaborative Intention, das die Bereitschaft der Nutzer misst, aktiv zur Verbesserung von KI-Systemen beizutragen. Die empirische Validierung erfolgte mittels einer Crowdsourcing-Kampagne (N=400) auf der Prolific-Plattform anhand einer App zur Schadensschätzung bei Autounfällen. Die Ergebnisse zeigen, dass XAI-bezogene Faktoren einen starken positiven Effekt auf Nutzungsabsicht, wahrgenommene Nützlichkeit und Benutzerfreundlichkeit haben. Zudem besteht ein signifikanter Zusammenhang zwischen Nutzungsabsicht und Kollaborationsbereitschaft, was die erfolgreiche Implementierung von Human-in-the-Loop-Ansätzen in Endnutzeranwendungen unterstützt.

Der Artikel (48 Zitierungen) ist für die Forschung zur Nutzerakzeptanz von KI-Systemen besonders relevant, da er erstmals ein validiertes Messinstrument für die spezifischen Herausforderungen der Mensch-KI-Kollaboration bereitstellt. Eine Limitation besteht darin, dass das Modell bisher nur an einem spezifischen Anwendungsszenario (Versicherungs-App) getestet wurde; weitere Validierungen in anderen Kontexten stehen noch aus.



\newpage
\fullcite{davis_perceived_1989}

\vspace{1em}
Diese wegweisende Studie entwickelt und validiert das Technology Acceptance Model (TAM), das zwei zentrale Konstrukte zur Vorhersage der Nutzerakzeptanz von Informationstechnologie einführt: Perceived Usefulness (wahrgenommene Nützlichkeit), definiert als der Grad, zu dem eine Person glaubt, dass die Nutzung eines Systems ihre Arbeitsleistung verbessert, und Perceived Ease of Use (wahrgenommene Benutzerfreundlichkeit), der Grad, zu dem die Nutzung als aufwandsfrei empfunden wird. Davis entwickelte Multi-Item-Skalen, die in zwei empirischen Studien mit insgesamt 152 Nutzern und vier Anwendungsprogrammen getestet wurden. Die resultierenden 6-Item-Skalen erreichten exzellente Reliabilitätswerte ($\alpha$=.98 für Usefulness, $\alpha$=.94 für Ease of Use) sowie hohe konvergente, diskriminante und faktorielle Validität. Die Ergebnisse zeigen, dass Perceived Usefulness signifikant stärker mit der Systemnutzung korreliert (r=.63 bzw. r=.85) als Perceived Ease of Use (r=.45 bzw. r=.59). Regressionsanalysen legen nahe, dass Ease of Use primär als kausaler Antezedent von Usefulness wirkt und nicht als paralleler, direkter Prädiktor der Nutzung.
Der Artikel (265 Zitierungen) gilt als Grundlagenwerk der Technologieakzeptanzforschung und hat das TAM als eines der meistzitierten Modelle in der Wirtschaftsinformatik etabliert. Die validierten Skalen werden bis heute in zahlreichen Studien zur Nutzerakzeptanz eingesetzt. Eine Limitation besteht darin, dass die Nutzungsmessung auf Selbstberichten basiert und keine objektiven Nutzungsdaten erhoben wurden.

\vspace{2em}

\newpage
\fullcite{levin_all_1998}

\vspace{1em}
Diese einflussreiche Metaanalyse entwickelt eine Typologie zur Unterscheidung dreier grundlegend verschiedener Arten von Valenz-Framing-Effekten, die in der Literatur häufig vermischt wurden: Risky Choice Framing beeinflusst die Risikobereitschaft bei Entscheidungen zwischen sicheren und unsicheren Optionen (z.B. Tversky \& Kahnemans „Asian Disease Problem"); Attribute Framing beeinflusst die Bewertung einzelner Objekteigenschaften (z.B. „75\% mager" vs. „25\% fett" bei Hackfleisch); und Goal Framing beeinflusst die Überzeugungskraft von Botschaften durch Betonung von Gewinnen oder Verlusten (z.B. Brustselbstuntersuchung). Die Autoren analysieren systematisch die unterschiedlichen operationalen Definitionen, abhängigen Variablen und zugrundeliegenden psychologischen Mechanismen jedes Framing-Typs. Während Risky Choice Framing durch Prospect Theory erklärt wird, basiert Attribute Framing auf assoziativem Encoding in valenzkongruenten Gedächtnisstrukturen, und Goal Framing wird durch Verlustaversion und Negativitätsbias erklärt. Die Typologie löst scheinbare Widersprüche in der Literatur auf, etwa warum positive Frames bei Attributbewertungen vorteilhaft sind, während negative Frames bei Überzeugungskommunikation wirksamer sein können.
Der Artikel (1.899 Zitierungen) gilt als Standardwerk zur konzeptuellen Differenzierung von Framing-Effekten und hat die methodische Präzision in der Entscheidungsforschung massgeblich beeinflusst. Die Typologie wird bis heute als Orientierungsrahmen für die Einordnung und Gestaltung von Framing-Studien verwendet. Eine Limitation besteht darin, dass die Grenzen zwischen den drei Framing-Typen in komplexen Realsituationen nicht immer trennscharf sind und Mischformen auftreten können.

\vspace{1em}


\newpage
\fullcite{freling_when_2014}

\vspace{1em}
Diese Studie hinterfragt die etablierte Annahme, dass positive Attribut-Frames grundsätzlich wirksamer sind als negative. Die Autoren wenden die Construal Level Theory (CLT) auf die Attribut-Framing-Forschung an und führen zunächst eine Meta-Analyse von 107 publizierten Artikeln (N=88.326 Beobachtungen) durch. Zentrale Erkenntnis ist, dass Framing-Effekte am stärksten sind, wenn Kongruenz zwischen dem Konstrual-Level der Botschaft (abstrakt vs. konkret) und der psychologischen Distanz des Empfängers zum geframten Ereignis besteht. Ein Folgeexperiment (N=100) bestätigt, dass nicht die Valenz allein, sondern die Übereinstimmung zwischen Konstrual-Level und psychologischer Distanz (temporal, hypothetisch, affektiv, informationell, sozial) die Framing-Effekte treibt. Die Ergebnisse zeigen: Positive Frames wirken besser bei psychologisch distanten Ereignissen, während negative Frames bei psychologisch nahen Ereignissen effektiver sein können.

Der Artikel (52 Zitierungen) leistet einen wichtigen Beitrag zur Framing-Forschung, indem er den Fokus von der reinen Botschaftsgestaltung auf die komplexe Beziehung zwischen Botschaft und Empfänger verlagert. Dies hat praktische Implikationen für die Gestaltung persuasiver Kommunikation in Marketing, Führung und Verhandlungen. Eine Limitation besteht darin, dass die Meta-Analyse auf publizierte Studien beschränkt ist, was potenzielle Publication-Bias-Effekte nicht vollständig ausschliesst.

\end{document}
