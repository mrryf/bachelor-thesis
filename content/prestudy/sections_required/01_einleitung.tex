\documentclass[../main_required.tex]{subfiles}
\begin{document}

% ============================================
% Section 1: Einleitung
% ============================================
\section{Einleitung}

Mit der Veröffentlichung von ChatGPT von OpenAI im Jahr 2022 \parencite{cunningham_how_2025} wurde eine technologische Wende eingeleitet. Bereits heute vereinfachen und verändern LLM-basierte Applikationen wie ChatGPT von OpenAI \parencite{openai_openai_2025}, Claude von Anthropic \parencite{anthropic_anthropic_2025} und Gemini von Google \parencite{google_inc_google_2025} viele Tätigkeiten des (Arbeits-)Lebens. Die rasante Verbreitung dieser Technologie zeigt sich eindrücklich: Innerhalb von nur sieben Monaten im Jahr 2025 konnte OpenAI seine Nutzerbasis von 350 auf über 700 Millionen wöchentlich aktive Nutzer steigern \parencite{cunningham_how_2025}.

Chatbots spielen im Alltag inzwischen in mehrfacher Hinsicht eine wichtige Rolle: Sie unterstützen bei der Informationsbeschaffung, geben praktische Anleitungen und bieten zum Beispiel Hilfe in der Programmierung sowie in Kreativprozessen. Dabei treten sie als tägliche Begleiter des Menschen auf: Ob durch eine bewusst durchgeführte Interaktion oder als ein im Hintergrund stattfindender, unbewusster Berührungspunkt \parencite{cunningham_how_2025}.

Mit ihrer relativ jungen (und öffentlichkeitswirksamen) Geschichte ist die generative künstliche Intelligenz, wie viele übergreifende technologischen Veränderungen, einem technologischen- und gesellschaftlichen Adoptionsprozess ausgesetzt. Einen theoretischen Erklärungsansatz dieses Adoptionsprozesses liefert Fred Davis 1989 mit seinem Werk «User acceptance of information systems: the technology acceptance model (TAM)». In seiner Arbeit legt Davis den Fokus auf die wahrgenommene Nützlichkeit («Perceived Usefulness») und die Einfachheit der Nutzung («Ease of Use»), woraus die Verhaltensintention («Behavourial Intention») abgeleitet wird \parencite{davis_perceived_1989}. Im Kontext von generativer KI, oder künstlicher Intelligenz im Allgemeinen, ist jedoch der Aspekt des Vertrauens in die Technologie von besonderer Bedeutung. Neben Nützlichkeit und Einfachheit stellt die Vertrauensfrage den Aspekt dar, ob künstlicher Intelligenz vertrauenswürdig ist. Sämtliche grossen Anbieter wie ChatGPT, Claude und Gemini weisen vor- sowie während der Nutzung ausdrücklich darauf hin, dass ihre Modelle und konsequenterweise KI-Assistenten die auf diesen Modellen basieren, fehlerhaft sein können. Diese Fehleranfälligkeit sowie zusätzliche Vorbehalte, wie die Angst vor Jobverlust, Bedenken hinsichtlich der Privatsphäre oder ethische Fragen \parencite{li_dimensions_2020}, erfordern die Integration und Erfassung von «Vertrauen» als eigenständiges Konstrukt in möglichen theoretischen Modellen.

% ============================================
% Subsection 1.1: Ausgangslage
% ============================================
\subsection{Ausgangslage}
Mit der Lancierung von Alva erweitert der Kanton Basel-Stadt sein bestehendes Informationsangebot um eine KI-gestützte Interaktionsform. Bei Alva handelt es sich um einen LLM-basierten Chatbot, der auf der Technologie von ChatGPT basiert und die Inhalte der Kantonswebsite bs.ch als Wissensbasis nutzt, um Fragen der Bevölkerung in natürlicher Konversationsform zu beantworten.

Die Einführung von Alva markiert für den Kanton Basel-Stadt einen bedeutsamen Schritt: Es handelt sich um eine der ersten KI-gestützten Lösungen dieser Art im kantonalen Kontext. Die gewonnenen Erkenntnisse aus diesem Pilotprojekt sollen als Grundlage für weitere KI-basierte Initiativen des Kantons dienen.

Aktuell verzeichnet Alva täglich rund 700 aktive Nutzer, die im Durchschnitt 1.4 Interaktionen mit dem digitalen Assistenten durchführen. Nebst der erwarteten Effizienzsteigerung bei der Informationsbeschaffung ist es von besonderem Interesse zu untersuchen, inwiefern das Vertrauen in die KI-Lösung die Nutzungsabsicht beeinflusst.

% ============================================
% Subsection 1.2: Zielsetzung
% ============================================
\subsection{Zielsetzung}
Die vorliegende Bachelorarbeit verfolgt zwei zentrale Ziele: Erstens soll empirisch untersucht werden, wie unterschiedliche Darstellungsformen von KI-Konfidenzwerten (positives vs. negatives Framing) das Nutzervertrauen in LLM-basierte Assistenzsysteme beeinflussen \parencite{levin_all_1998, kim_communicating_2022}. Zweitens soll die Anwendbarkeit des Artificial Intelligence Technology Acceptance Model (AI-TAM) im Kontext eines öffentlichen KI-Assistenten validiert werden, insbesondere hinsichtlich der Mediationsrolle von Vertrauen zwischen Transparenzkommunikation und Nutzungsintention \parencite{baroni_ai-tam_2022}.

% ============================================
% Subsection 1.3: Machbarkeit
% ============================================
\subsection{Machbarkeit}

Das vorliegende Forschungsdesign basiert auf einem experimentellen Testen verschiedener Konfidenz-Werte innerhalb eines chatbot-basierten Umfeldes. Das Experiment soll im Idealfall in einer tatsächlichen Chatbot-Interaktion stattfinden, anstelle einer fragebogenbasierten Stimulus-Darbietung.

Die von Basel-Stadt entwickelte Lösung «Alva» \parencite{alva_2025} dient als zentraler digitaler Assistent bei der Bedienung und Navigation der Website des Kantons Basel-Stadt \parencite{kanton_basel_stadt_2025}. Alva verfügt über die sämtlichen Inhalte der Kantonswebsite als Wissensbasis und ermöglicht es Nutzern, Informationen zu gewünschten Themen abzurufen. Das Abrufen von Informationen funktioniert themen- und bereichsübergreifend, was inhaltlich anspruchsvolle Themen und Prozesse in einfache Schritte herunterbricht und die benötigten Links und Dokumente als Referenzinformationen zusätzlich zur gelieferten Antwort auf die gestellte Anfrage bereitstellt. Alva zählt zum heutigen Zeitpunkt täglich rund 550 Nutzer mit durchschnittlich 1.4 Interaktionen pro Nutzer.

Nach initialen Unterhaltungen ist der Kanton Basel-Stadt einverstanden, das vorgesehene Experiment in der Live-Umgebung von Alva durchzuführen. Die anfallenden Arbeiten zur Integration werden zu je 50\% vom Auftraggeber Liip \parencite{liip_2025} und dem Kanton Basel-Stadt getragen. Die benötigte Stimulus-Konzeption und das Survey-Design obliegen in der Verantwortung des Studierenden.

\begin{table}[h!]
    \centering
    \caption{Meilensteine der Vorstudie und Bachelorarbeit}
    \label{tab:milestones}
    \begin{tabularx}{\textwidth}{@{}l l X@{}}
        \toprule
        \textbf{Meilenstein} & \textbf{Zeitraum} & \textbf{Beteiligte} \\
        \midrule
        Gespräch Machbarkeit intern & Juli 2025 & Liip \\
        Gespräch Machbarkeit extern & Oktober 2025 & Kanton Basel-Stadt \\
        Entwicklung Anforderungen (Logik \& Userflow) & Oktober 2025 & Studierender \\
        Schätzung benötigter Arbeiten & November 2025 & Product Owner, Frontend Developer \\
        Kommunikation Investment extern & November 2025 & Kanton Basel-Stadt \\
        Übereinkunft Investment-Teilung & November 2025 & Liip, Kanton Basel-Stadt \\
        \bottomrule
    \end{tabularx}
\end{table}


% ============================================
% Subsection 1.4: Arbeitsplan
% ============================================
\begin{landscape}
\thispagestyle{empty}
% Page number at top-right of landscape page
\vspace*{-2\baselineskip}\hfill\thepage\par\vspace{\baselineskip}
\subsection{Arbeitsplan}

Der folgende Arbeitsplan zeigt die zeitliche Planung der Vorstudie.

\begin{figure}[h!]
    \centering
    \resizebox{\linewidth}{!}{ % Fit to landscape width
    \begin{ganttchart}[
        hgrid,
        vgrid={*1{dotted, black!20}},
        x unit=0.4cm, % Squeezed timeline to give more space to labels
        y unit chart=1.5cm, 
        time slot format=isodate,
        calendar week text={\currentweek},
        % APA Style: Clean, professional, grayscale/muted
        bar/.append style={fill=black!60, draw=none},
        bar height=0.7,
        bar label font=\Large\bfseries, % Bigger and bold labels
        group/.append style={draw=black, fill=black!80},
        group label font=\Large\bfseries, % Bigger group labels
        group right shift=0,
        group top shift=0.6,
        group height=.3,
        group peaks width={0.2},
        milestone/.append style={fill=black, draw=none},
        milestone height=0.5,
        milestone label font=\Large\itshape % Bigger milestone labels
    ]{2025-09-01}{2025-12-14}
    \gantttitlecalendar{year, month=name, week} \\
    
    % Vorstudie
    \ganttgroup{Vorstudie (KW 36 - 49, 2025)}{2025-09-01}{2025-12-07} \\
    \ganttbar{Themeneingabe}{2025-09-01}{2025-09-28} \\ % KW 36-39
    \ganttbar{Vorbereitung Vorstudiengespräch}{2025-09-29}{2025-10-05} \\ % KW 40
    \ganttbar{Durchführung Vorstudiengespräch}{2025-10-06}{2025-10-12} \\ % KW 41
    \ganttbar{Zoom Call Vorstudie/BA}{2025-11-10}{2025-11-16} \\ % KW 46
    \ganttbar{Zoom Call Methodisches Vorgehen}{2025-11-17}{2025-11-23} \\ % KW 47
    \ganttbar{Literaturecherche}{2025-11-17}{2025-12-07} \\ % KW 47-49
    \ganttbar{Verfassen der Vorstudie}{2025-11-24}{2025-12-07} \\ % KW 48-49
    \ganttmilestone{Abgabe der Vorstudie}{2025-12-05} 

    \end{ganttchart}
    }
    \vspace{0.5cm} % Fix caption overlap
    \caption{Gantt-Chart Arbeitsplan}
    \label{fig:gantt_chart}
\end{figure}
\end{landscape}

\end{document}
