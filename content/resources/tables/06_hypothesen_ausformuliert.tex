\newpage
\begin{table}[H]
    \centering
    \small
    \caption{Ausformulierte Hypothesen}
    \label{tab:hypotheses-full}
    \begin{tabularx}{\textwidth}{l l X}
        \toprule
        \textbf{Bereich} & \textbf{Nr.} & \textbf{Hypothese} \\
        \midrule
        \multicolumn{3}{l}{\textbf{Framing-Hypothesen}} \\
        \midrule
        Framing & H1a & Die Darstellung als Sicherheit (positiver Frame) führt zu einem höheren Vertrauen in künstliche Intelligenz als die Kontrollbedingung ohne Konfidenz-Darstellung. \\
         & H1b & Die Darstellung als Unsicherheit (negativer Frame) führt zu einem niedrigeren Vertrauen in künstliche Intelligenz als die Kontrollbedingung ohne Konfidenz-Darstellung. \\
        \midrule
        \multicolumn{3}{l}{\textbf{AI-TAM-Hypothesen}} \\
        \midrule
        AI-TAM & H2 & Vertrauen in künstliche Intelligenz hat einen positiven Einfluss auf die wahrgenommene Nützlichkeit (PUF). \\
         & H3 & Vertrauen in künstliche Intelligenz hat einen positiven Einfluss auf die wahrgenommene Einfachheit in der Nutzung (EOU). \\
        \midrule
        \multicolumn{3}{l}{\textbf{TAM-Hypothesen}} \\
        \midrule
        TAM & H4 & Die wahrgenommene Nützlichkeit (PUF) hat einen positiven Einfluss auf die Nutzungsintention. \\
         & H5 & Die wahrgenommene Einfachheit in der Nutzung (EOU) hat einen positiven Einfluss auf die Nutzungsintention. \\
         & H6 & Die wahrgenommene Einfachheit in der Nutzung (EOU) hat einen positiven Einfluss auf die wahrgenommene Nützlichkeit (PUF). \\
         & H7 & Die Nutzungsintention hat einen positiven Einfluss auf die Kollaborationsintention. \\
         & H8 & Die Vertrautheit mit Technologie (FAM-TEC) hat einen positiven Einfluss auf die wahrgenommene Nützlichkeit (PUF). \\
        \bottomrule
    \end{tabularx}
\end{table}
