\paragraph{Validierung durch die BumpOut-Studie \parencite{baroni_ai-tam_2022}}
Das AI-TAM wurde mit der Anwendung «BumpOut» validiert, einer KI-gestützten App zur Schadensmeldung bei Autounfällen. Die Studie umfasste 400 Teilnehmende in zwei Crowdsourcing-Kampagnen unter unterschiedlichen experimentellen Bedingungen: einer fehlerfreien KI versus einer teilweise fehlerhaften KI. Die App analysiert dabei Schadensbilder automatisch, wobei die Nutzer die von der KI getroffenen Identifikationen bestätigen oder korrigieren können. Die Ergebnisse zeigten hohe Werte für die wahrgenommene Nützlichkeit und Benutzerfreundlichkeit, während die Funktionsfähigkeit der KI nur einen minimalen Einfluss hatte. Nennenswert war die starke Korrelation zwischen der Nutzungsabsicht und der Kollaborationsabsicht. Dies bestätigt, dass Nutzer, die die App verwenden, auch bereit sind, zur Verbesserung der KI beizutragen. Das AI-TAM eignet sich daher auch für die Untersuchung der Akzeptanz von Large Language Models (LLMs), da diese Systeme die gleichen kritischen Charakteristika aufweisen: probabilistische Ausgaben, inhärente Unsicherheit und die Notwendigkeit von Nutzervertrauen. LLMs werden zunehmend als kollaborative Partner wahrgenommen, deren Ergebnisse oft Nutzerfeedback erfordern. Insbesondere die XAI-Konstrukte sind hier relevant, da Nutzer nachvollziehen müssen, warum ein LLM eine bestimmte Antwort generiert.
