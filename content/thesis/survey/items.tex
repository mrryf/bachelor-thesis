\documentclass[../main.tex]{subfiles}
\begin{document}

\section{Survey Measurement Items}

This annex presents all measurement items used in the empirical survey, organized by theoretical construct. Each table displays the construct name, original items from the literature, adapted items for this study, scaling method, and source references. The items were measured using 5-point Likert scales (ranging from 1 = \textit{strongly disagree} to 5 = \textit{strongly agree}) unless otherwise noted. All items were presented to participants in German.

The constructs measured in this study include: Treatment Check, Perceived Usefulness (PUF), Ease of Use (EOU), Behavioral Intention (BI), Explainable AI Trust (XAIT), Familiarity with Technology (FAM-TEC), and Collaborative Intention (CI). These constructs form the theoretical foundation of the research model and were selected based on their relevance to understanding user acceptance and trust in AI systems.

\subsection{Treatment Check}
\label{subsec:treatment-check-items}
The treatment check item was used to verify that participants noticed the experimental manipulation regarding answer certainty/uncertainty indicators in the AI system responses.

\begin{table}[ht]
\centering
\begin{threeparttable}
\caption{Items for Construct: Treatment_Check}
\label{tab:Treatment_Check}
\begin{tabularx}{\textwidth}{l X X l l}
\toprule
\textbf{Construct} & \textbf{Original Item} & \textbf{Adapted Item} & \textbf{Scaling} & \textbf{Source} \\
\midrule
Treatment Check & Ist Ihnen die Einschätzung zur Antwortsicherheit bzw. Antwortunsicherheit bei Chat-Antworten aufgefallen? & Haben Sie die Bewertung zur Antwortsicherheit- und Unsicherheit in den Antworten wahrgenommen? & j/n (boolean) & - \\
\bottomrule
\end{tabularx}
\end{threeparttable}
\end{table}

\subsection{Perceived Usefulness (PUF)}
\label{subsec:puf-items}
Perceived Usefulness measures the degree to which participants believe that using the AI system (Alva) would enhance their work performance and productivity.

\begin{table}[ht]
\centering
\begin{threeparttable}
\caption{Items for Construct: PUF}
\label{tab:PUF}
\begin{tabularx}{\textwidth}{l X X l l}
\toprule
\textbf{Construct} & \textbf{Original Item} & \textbf{Adapted Item} & \textbf{Scaling} & \textbf{Source} \\
\midrule
PUF & Der Einsatz von künstlicher Intelligenz verbessert meine Arbeit & Der Einsatz von Alva verbessert meine Arbeit. & 1-5 Likert-Skala & Ibrahim et al. (2025) \\
 & Der Einsatz von künstlicher Intelligenz steigert meine Effektivität & Der Einsatz von Alva steigert meine Effektivität. & 1-5 Likert-Skala & Ibrahim et al. (2025) \\
 & Durch den Einsatz von künstlicher Intelligenz bin ich produktiver & Durch den Einsatz von Alva bin ich produktiver. & 1-5 Likert-Skala & Ibrahim et al. (2025) \\
 & Künstliche Intelligenz ist bei meiner Arbeit ein nützliches Tool. & Alva ist bei meiner Aufgabe ein nützliches Tool & 1-5 Likert-Skala & Ibrahim et al. (2025) \\
\bottomrule
\end{tabularx}
\end{threeparttable}
\end{table}

\subsection{Ease of Use (EOU)}
\label{subsec:eou-items}
Ease of Use assesses how effortlessly participants can interact with and utilize the AI system without requiring extensive cognitive effort.

\begin{table}[ht]
\centering
\begin{threeparttable}
\caption{Items for Construct: EOU}
\label{tab:EOU}
\begin{tabularx}{\textwidth}{l X X l l}
\toprule
\textbf{Construct} & \textbf{Original Item} & \textbf{Adapted Item} & \textbf{Scaling} & \textbf{Source} \\
\midrule
EOU & Künstliche Intelligenz so anzuwenden &  wie ich es brauche &  fällt mir leicht. & Alva so anzuwenden \\
 & Beim Einsatz von künstlicher Intelligenz muss ich nicht lange nachdenken. & Beim Einsatz von Alva muss ich nicht lange nachdenken & 1-5 Likert-Skala & Ibrahim et al. (2025) \\
 & Ich gehe mit künstlicher Intelligenz geradlinig und klar um. & Ich gehe mit Alva geradlinig und klar um. & 1-5 Likert-Skala & Ibrahim et al. (2025) \\
 & Künstliche Intelligenz nutze ich ganz unbekümmert. & Alva nutze ich ganz unbekümmert & 1-5 Likert-Skala & Ibrahim et al. (2025) \\
\bottomrule
\end{tabularx}
\end{threeparttable}
\end{table}

\subsection{Behavioral Intention (BI)}
\label{subsec:bi-items}
Behavioral Intention captures participants' willingness and intention to use AI systems in future contexts and applications.

\begin{table}[ht]
\centering
\begin{threeparttable}
\caption{Items for Construct: BI}
\label{tab:BI}
\begin{tabularx}{\textwidth}{l X X l l}
\toprule
\textbf{Construct} & \textbf{Original Item} & \textbf{Adapted Item} & \textbf{Scaling} & \textbf{Source} \\
\midrule
BI & Ich will künstliche Intelligenz in der Zukunft nutzen. & - & 1-5 Likert-Skala & Ibrahim et al. (2025) \\
 & Ich bemühe mich &  die Nutzung künstlicher Intelligenz auf das sinnvolle zu begrenzen. & - & 1-5 Likert-Skala \\
 & Künstliche Intelligenz soll bei mir in der Zukunft häufiger zum Einsatz kommen. & - & 1-5 Likert-Skala & Ibrahim et al. (2025) \\
 & Die Nutzung künstlicher Intelligenz werde ich künftig noch auf weitere Bereiche ausdehnen. & - & 1-5 Likert-Skala & Ibrahim et al. (2025) \\
\bottomrule
\end{tabularx}
\end{threeparttable}
\end{table}

\subsection{Explainable AI Trust (XAIT)}
\label{subsec:xait-items}
Explainable AI Trust measures participants' general attitudes toward AI, including trust, fear, and beliefs about AI's impact on society and employment.

\begin{table}[ht]
\centering
\begin{threeparttable}
\caption{Items for Construct: XAIT}
\label{tab:XAIT}
\begin{tabularx}{\textwidth}{l X X l l}
\toprule
\textbf{Construct} & \textbf{Original Item} & \textbf{Adapted Item} & \textbf{Scaling} & \textbf{Source} \\
\midrule
XAIT & Ich habe Angst vor künstlicher Intelligenz. & - & 1-5 Likert-Skala & Ibrahim et al. (2025) \\
 & Ich vertraue künstlicher Intelligenz. & - & 1-5 Likert-Skala & Ibrahim et al. (2025) \\
 & Künstliche Intelligenz wird die Menschheit zerstören. & - & 1-5 Likert-Skala & Ibrahim et al. (2025) \\
 & Künstliche Intelligenz wird eine Bereicherung für die Menschheit sein. & - & 1-5 Likert-Skala & Ibrahim et al. (2025) \\
 & Künstliche Intelligenz wird für viel Arbeitslosigkeit sorgen. & - & 1-5 Likert-Skala & Ibrahim et al. (2025) \\
\bottomrule
\end{tabularx}
\end{threeparttable}
\end{table}

\subsection{Familiarity with Technology (FAM-TEC)}
\label{subsec:fam-tec-items}
Familiarity with Technology assesses participants' prior knowledge and experience with generative AI systems such as ChatGPT, Copilot, or Gemini.

\begin{table}[ht]
\centering
\begin{threeparttable}
\caption{Items for Construct: FAM-TEC}
\label{tab:FAM-TEC}
\begin{tabularx}{\textwidth}{l X X l l}
\toprule
\textbf{Construct} & \textbf{Original Item} & \textbf{Adapted Item} & \textbf{Scaling} & \textbf{Source} \\
\midrule
FAM-TEC & I know a lot about Gen-AI & Ich weiss viel über KI-Systeme wie ChatGPT &  Copilot oder Gemini & 1-5 Likert-Skala \\
 & I am familiar with Gen-AI & Ich bin vertraut mit KI-Systemen wie ChatGPT &  Copilot oder Gemini & 1-5 Likert-Skala \\
 & I have much knowledge about Gen-AI & Ich habe viel Wissen über KI-Systeme wie ChatGPT &  Copilot oder Gemini & 1-5 Likert-Skala \\
 & I know how to interact with Gen-AI & Ich weiss &  wie man mit KI-Systemen wie ChatGPT &  Copilot oder Gemini interagiert \\
\bottomrule
\end{tabularx}
\end{threeparttable}
\end{table}

\subsection{Collaborative Intention (CI)}
\label{subsec:ci-items}
Collaborative Intention evaluates participants' perceptions of the AI system's accuracy, dependability, consistency, and efficiency in collaborative interactions.

\begin{table}[ht]
\centering
\begin{threeparttable}
\caption{Items for Construct: CI}
\label{tab:CI}
\begin{tabularx}{\textwidth}{l X X l l}
\toprule
\textbf{Construct} & \textbf{Original Item} & \textbf{Adapted Item} & \textbf{Scaling} & \textbf{Source} \\
\midrule
CI & The system is accurate & Alva ist genau in Chat-Antworten & 1-5 Likert-Skala & Grassi et al. (2022) \\
 & The system is dependable & Alvas Chat-Antworten sind zuverlässig & 1-5 Likert-Skala & Grassi et al. (2022) \\
 & The system makes few errors & Alva macht wenig Fehler & 1-5 Likert-Skala & Grassi et al. (2022) \\
 & The interaction with the system is consistent & Die Interaktion mit Alva ist konsistent & 1-5 Likert-Skala & Grassi et al. (2022) \\
 & The interaction with the system is efficient & Die Interaktion mit Alva ist effizient & 1-5 Likert-Skala & Grassi et al. (2022) \\
\bottomrule
\end{tabularx}
\end{threeparttable}
\end{table}

\end{document}
